\documentclass[12pt]{article}

\include{preamble}

\newtoggle{professormode}
\toggletrue{professormode} %STUDENTS: DELETE or COMMENT this line



\title{MATH 340/640 Fall \the\year~ Homework \#4}

\author{Professor Adam Kapelner} %STUDENTS: write your name here

\iftoggle{professormode}{
\date{Due by email 11:59PM on the date found on the homepage \\ \vspace{0.5cm} \small (this document last updated \today ~at \currenttime)}
}

\renewcommand{\abstractname}{Instructions and Philosophy}

\begin{document}
\maketitle

\iftoggle{professormode}{
\begin{abstract}
The path to success in this class is to do many problems. Unlike other courses, exclusively doing reading(s) will not help. Coming to lecture is akin to watching workout videos; thinking about and solving problems on your own is the actual ``working out.''  Feel free to \qu{work out} with others; \textbf{I want you to work on this in groups.}

Reading is still \textit{required}. For this homework set, review Math 241 concerning the exponential.

The problems below are color coded: \ingreen{green} problems are considered \textit{easy} and marked \qu{[easy]}; \inorange{yellow} problems are considered \textit{intermediate} and marked \qu{[harder]}, \inred{red} problems are considered \textit{difficult} and marked \qu{[difficult]} and \inpurple{purple} problems are extra credit. The \textit{easy} problems are intended to be ``giveaways'' if you went to class. Do as much as you can of the others; I expect you to at least attempt the \textit{difficult} problems. \qu{[MA]} are for those registered for 621 and extra credit otherwise.

This homework is worth 100 points but the point distribution will not be determined until after the due date. See syllabus for the policy on late homework.

Up to 5 points are given as a bonus if the homework is typed using \LaTeX. Links to instaling \LaTeX~and program for compiling \LaTeX~is found on the syllabus. You are encouraged to use \url{overleaf.com}. If you are handing in homework this way, read the comments in the code; there are two lines to comment out and you should replace my name with yours and write your section. The easiest way to use overleaf is to copy the raw text from hwxx.tex and preamble.tex into two new overleaf tex files with the same name. If you are asked to make drawings, you can take a picture of your handwritten drawing and insert them as figures or leave space using the \qu{$\backslash$vspace} command and draw them in after printing or attach them stapled.

The document is available with spaces for you to write your answers. If not using \LaTeX, print this document and write in your answers. I do not accept homeworks which are \textit{not} on this printout. Keep this first page printed for your records.

\end{abstract}

\thispagestyle{empty}
\vspace{1cm}
NAME: \line(1,0){380}
\clearpage
}



\problem{These exercises will continue to introduce the Multinomial distribution. }


\begin{enumerate}


\easysubproblem{Consider the following rv for the remaining exercises in this problem:\\ $\X \sim \multinomial{17}{\bracks{0.1~0.2~0.3~0.4}^\top}$. What is $\dime{\X}$?}\spc{0}


\intermediatesubproblem{Find $\var{\X}$.}\spc{4}

\intermediatesubproblem{If $x_1 = 1$, what is the JMF of the remaining rv's?}\spc{4.5}

\intermediatesubproblem{If $x_1 = 1$ and $x_2 = 6$, what is the JMF of the remaining rv's?}\spc{4.5}


\easysubproblem{If $x_1 = 1$, $x_2 = 6$ and $x_3 = 3$, how is the remaining rv distributed?}\spc{3}

\end{enumerate}




\problem{These exercises will give you practice with transformations of discrete r.v.'s.}


\begin{enumerate}

\easysubproblem{Let $X \sim \binomial{n}{p}$. Find the PMF of $Y = \natlog{X + 1}$.}\spc{3}

\easysubproblem{Let $X \sim \binomial{n}{p}$. Find the PMF of $Y = X^3$.}\spc{3}

\intermediatesubproblem{Show that for any r.v. $X$ (discrete or continuous), if $Y = aX + b$, then $F_Y(y) = F_X\parens{\frac{y - b}{a}}$.}\spc{2}

\intermediatesubproblem{Let $X \sim \negbin{k}{p}$. Find the PMF of $Y = X^2$. Is $g(X)$ monotonic? Does that matter for this r.v.?}\spc{2}

\hardsubproblem{Let $X \sim \binomial{n}{p}$ where $n$ is an even number. Find an expression for the PMF of $Y = \text{mod}(X, 2)$.}\spc{5}

\end{enumerate}


\problem{These exercises will give you practice with transformations of continuous r.v.'s.}


\begin{enumerate}

\inthenotessubproblem{Let $g$ be a strictly decreasing function and $X$ be a continuous rv and $Y = g(X)$. Find a formula for the PDF of $Y$. Justify each step.}\spc{9}

\inthenotessubproblem{Let $g(x) = ax + b$, $X$ be a continuous rv and $Y = g(X)$. Find a formula for the PDF of $Y$. Do this step-by-step (i.e., first find the inverse function).}\spc{4}



\intermediatesubproblem{Let $X \sim \exponential{\lambda}$. Show that $Y = aX$ is an exponential rv and find its paramter. Use the transformation formula (not ch.f.'s).}\spc{5}


\hardsubproblem{Let $X \sim \text{Logistic}(0,1)$. Find the PDF of $Y = g(X) = \oneover{1 + e^{-X}}$. If this is a brand name r.v., mark it so and include its parameter values.}\spc{5}


\intermediatesubproblem{Let $X \sim \exponential{\lambda}$. Find the PDF of $Y = g(X) = ke^X$ where $k>0$. This will be a brand name r.v., so mark it so and include its parameter values.}\spc{8}

\intermediatesubproblem{Let $X \sim \exponential{\lambda}$. Find the PDF of $Y = g(X) = \natlog{X}$.}\spc{5}


\hardsubproblem{If $X \sim \exponential{\lambda}$ then show that $Y = X^\beta \sim \text{Weibull}$ where $\beta > 0$. Find the resulting Weibull's parameters in terms of the parameterization we learned in class.}\spc{7}

\extracreditsubproblem{Let $X \sim \exponential{\lambda}$. Find the PDF of $Y = g(X) = \sin{X}$. Don't attempt this unless you have extra time.}\spc{6}


\easysubproblem{Rederive the $X \sim \text{Laplace}(0, 1)$ r.v. model by taking the difference of two standard exponential r.v.'s.}\spc{8}


\easysubproblem{Show that $\mathcal{E} \sim \text{Laplace}(0, \sigma)$ satisfies the three conditions of the definition of an \qu{error distribution}.}\spc{6}

\end{enumerate}


\problem{These exercises will give you practice with the gamma function.}


\begin{enumerate}

\inthenotessubproblem{Write the definition of $\Gamma\parens{x}$.}\spc{0.5}

\hardsubproblem{Prove $\Gamma\parens{k + 1} = k \Gamma\parens{k}$ for $k > 0$.}\spc{6}


\intermediatesubproblem{Write the definition of $Q\parens{x,a}$ without using the gamma function.}\spc{3}


\intermediatesubproblem{If $0 < a < b < \infty$, find an integral expression for $\Gamma\parens{x, b} - \gamma\parens{x, a}$.}\spc{1.5}


%\easysubproblem{For $a,c \in (0, \infty)$, prove $\int_a^\infty t^{x-1} e^{-ct} dt = \frac{\Gamma\parens{x, ac}}{c^x}$.}\spc{4}


\intermediatesubproblem{Let $X \sim \text{Gamma}(\alpha, \beta)$. Prove the Humpty Dumpty identity. How do you think the Gamma rv got its name?}\spc{4}



\inthenotessubproblem{Write the PDF's and parameter spaces of both the gamma rv and the Erlang rv model. Explain how the former \qu{upgrades} the latter.}\spc{3}

\end{enumerate}

\end{document}
